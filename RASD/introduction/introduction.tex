\providecommand{\rasd}{..}
\documentclass[../RASD.tex]{subfiles}

\begin{document}
    \chapter{Introduction}\label{ch:introduction}
    \section{Purpose}\label{sec:purpose}
        \subsection{General Purpose}\label{subsec:general-purpose}
        The purpose of the Requirement Analysis and Specification Document (RASD) is to give a clear and non-ambiguos analysis of SafeStreets,
        an application that will be implemented, describing every aspects of it like functional/non-functional requirements, constraints,
        domain assumptions, providing use cases and scenerios of the external world.
        Moreover, a more formal analysis of some relevant functions of the system will be provided using Alloy,
        a declarative specification language for expressing complex structural constraints and behavior in a software system.

        The application will allow every user to take pictures of a violation and send them with the date, time, position,
        a scan of the license plate and an optional brief description to SafeStreets, that will give access to these information
        to both citizens and police officers, with different levels of visibility.

        If the municipality provides information about the accidents that occur in its territory, the system will be able to cross these data with its own,
        to define dangerous areas and suggest possible interventions.
        One of the objectives of SafeStreets is to help authorities to detect infringements;
        in order to do that, it will ensure that data will not be tempered, to prevent malicious corruptions by offenders.
        The authorities that will use SafeStreets to make fines will have the possibility to mark the violation on the system,
        to both provide a feedback to the citizens and allow SafeStreets to compile statistics.

    \subsection{Goals}\label{subsec:goals}
        SafeStreets is a service provided to people to notify both other people and authorities about traffic violations.
        There are two different classes of users to whom the software is addressed: citizens and authorities, which have two different ways to interact with the application.
        In order to perform correctly, the software-to-be will have to grant that some services will be guaranteed.
        Below is given a list of all the goal of the software-to-be:
        \begin{enumerate}
            \goal{1} The application will allow users to upload pictures of the traffic violations, including pictures, date, time, classification, and optionally a textual description.
            \goal{2} The application will allow users to view a map of the violations registered in the last day.
            \goal{3} The application will allow users to see the violations uploaded by other users;
            however, citizens won’t be able to see sensitive pictures, while authorities will be allowed.
            \goal{4} Users will be able to query the system to get all the reported violations that correspond to some temporal, geographical or categorical parameters.
            \goal{5} Users can give a feedback about violations uploaded by other users to SafeStreets.
            \goal{6} If SafeStreets can get the information about accidents by the municipality, it will give the possibility to merge them with its data to identify potentially unsafe areas.
            \goal{7}If SafeStreets can get the information about accidents by the municipality, it will give the possibility to merge them with its data to suggest possible interventions.
            \goal{8}The system will use the information about violations, accidents and fines that will collect by both users and authorities to build statistics.
        \end{enumerate}
    \newpage
    \section{Scope}\label{sec:scope}
        \subsection{World, Machine and Shared phenomena}\label{subsec:world,-machine-and-shared-phenomena}
        According to \textit{The World and the Machine} we can divide every system into two parts:
        \begin{itemize}
            \item The \textbf{machine}, which is the portion of system to be developed;
            \item The \textbf{world}, which is the portion of the real-world affected by the machine.
        \end{itemize}
        As a consequence we can classify phenomena in three different types:
        \begin{itemize}
            \item \textbf{World phenomena}: phenomena that the machine cannot observe;
            \item \textbf{Machine phenomena}: phenomena located entirely in the machine;
            \item \textbf{Shared phenomena}: phenomena that can be controlled by the world and observed by the machine or controlled by the machine and observed by the world;
        \end{itemize}
        Below we give an analysis of the three phenomenas described above:
        \begin{itemize}
            \item \textbf{World Phenomena}
            \begin{itemize}
                \item Users recognize the type of violation that occurred;
                \item People commit violations;
                \item People perform incidents;
                \item Police emits traffic tickets.
            \end{itemize}
            \item \textbf{Machine Phenomena}
            \begin{itemize}
                \item Machine manages database queries;
                \item Machine stores information (users and pictures with data)
                \\ on the database
                \item Machine manages interfaces with external software.
            \end{itemize}
            \item \textbf{Shared Phenomena}
            \begin{itemize}
                \item Users have to take and upload the pictures of the violations;
                \item Users query the system about violations;
                \item The machine gets the information about the accidents by the local municipality;
                \item The machine crosses information about accidents and violations to submit possible interventions, that can be both considered or not;
                \item The municipality can mark reported violations as fined, to allow the system to perform statistics about effectiveness of SafeStreets.
            \end{itemize}
        \end{itemize}

        \newpage
        %%%%%%%%%
        \begin{center}
            \begin{tabular}{ ||c|c|c|| }

                \hline
                \textbf{Phenomenon} & \textbf{Shared} & \textbf{Who controls it} \\ \hline
                Users recognize the type of violation that occurred & N & W \\ \hline
                People commit violations & N & W\\ \hline
                People perform accidents & N & W\\ \hline
                Police emits traffic tickets & N & W\\ \hline
                Machine manages database queries & N & M\\ \hline
                \makecell{Machine stores information (users and pictures with data)
                \\ on the database} & N & M\\ \hline
                Machine manages interfaces with external software & N & M\\ \hline
                Users have to take and upload the pictures of the violations & Y & W\\ \hline
                Users query the system to get violations & Y & W\\ \hline
                \makecell{The machine gets the information about the accidents
                \\from the local municipality} & Y & M\\ \hline
                \makecell{Machine crosses information about accidents
                \\and violations to submit possible interventions,
                \\that can be both considered or not} & Y & M\\ \hline
                \makecell{The municipality can mark reported violations as fined, \\ to allow the system
                to perform statistics about \\ the effectiveness of SafeStreets.} & Y & W\\

                \hline
            \end{tabular}
        \end{center}
        %%%%%%%%%
    \newpage
    \section{Definitions, Acronyms, Abbreviations}\label{sec:definitions,-acronyms,-abbreviations}
        \subsection{Definitions}\label{subsec:definitions}
        \begin{itemize}
            \item \textbf{User:} is a general customer that use the application. It is used to refer to both an authority and a ciziten.
            \item \textbf{Citizen:} is the basic customer of the application. He can upload violations and query the system to get statistics on a selected area.
            \item \textbf{Authority:} advanced customer of the application, is a registered user that is qualified to make fines and view sensitive information. Must verify himself during the registration.
            \item \textbf{Municipality:} is always intended as the local municipality. Every municipality provide and receives information exactly and only about its own jurisdiction.
            \item \textbf{Violation:} is a general infringement reported by a user.
            \item  \textbf{Dossier:} is the instance of a violation on the system, that includes the pictures, position, classification, textual specification and fine mark.
        \end{itemize}

        \subsection{Acronyms}\label{subsec:acronyms}
        \begin{itemize}
            \item RASD: Requirement Analysis and Specification Document
            \item LP: License Plate
            \item GPS: Global Positioning System
            \item API: Application Programming Interface
            \item UML: Unified Modeling Language
        \end{itemize}

        \subsection{Abbreviations}\label{subsec:abbreviations}
        \begin{itemize}
            \item $[Gn]$: n-goal.
            \item $[Dn]$: n-domain assumption.
            \item $[Rn]$: n-functional requirement.
            \item $[UCn]$: n-use case.
        \end{itemize}

    \section{Reference Documents}\label{sec:reference-documents}
    This document follows ISO/IEC/IEEE 29148:2011 and IEEE 830:1998 standard for software product specifications.
    All the specifications of this project have been given by Rossi and Di Nitto for the Software Engineering 2 Mandatory Project 2019-2020.

    \section{Revision history}\label{sec:revision-history}
    .

    \section{Document structure}\label{sec:document-structure}
    \textbf{Chapter 1: Introduction.} A general introduction to the goals, the phenomena and the scope of the system-to-be.
    It aims giving general but exhaustive information about what this document is going to explain.
    \\
    \textbf{Chapter 2: Overall description.} A general description of the product to be and its requirements.
    This section provides several information that are detailed explained in Section 3.
    \\
    \textbf{Chapter 3: Specific requirements.} All software requirements are explained using scenarios, use-case diagram and activity diagram.
    Non-functional and functional requirements are also mentioned.
    \\
    \textbf{Chapter 4: Alloy.} This section includes Alloy code that describes the model and checks whether it is consistent or not.
    \\
    \textbf{Chapter 5: Effort Spent.} A summary of the worked time by each member of the group.

\end{document}
