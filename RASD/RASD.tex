% PACKAGES
\documentclass[a4paper, hidelinks, 12pt]{report}
\usepackage[margin=1in]{geometry}
\usepackage{amsfonts,amsmath,amssymb}
\usepackage[none]{hyphenat}
\usepackage{fancyhdr}
\usepackage{graphicx}
\usepackage[nottoc,notlot,notlof]{tocbibind}
\usepackage{hyperref}
\usepackage{longtable}
\usepackage[utf8]{inputenc}
\usepackage{multirow}
\usepackage{booktabs}
\usepackage[font=footnotesize]{caption}
\usepackage[flushleft]{threeparttable}
\usepackage{relsize}
\usepackage[super,negative]{nth}
\usepackage{enumerate}
\usepackage{float}

\usepackage[dvipsnames]{xcolor}
\usepackage{listings}
%\usepackage{alloy-style}
%%%%%%%%%%%%
% DOC STYLES
%\makeatletter
\def\thickhrulefill{\leavevmode \leaders \hrule height 1ex \hfill \kern \z@}
\def\@makechapterhead#1{
	\vspace*{4\p@}
	{\parindent \z@ \centering \reset@font
		\thickhrulefill\quad
		\scshape \@chapapp{} \thechapter
		\quad \thickhrulefill
		\par\nobreak
		\vspace*{4\p@}
		\interlinepenalty\@M
		\hrule
		\vspace*{4\p@}
		\Huge \bfseries #1\par\nobreak
		\vspace*{4\p@}
		\hrule
		\vskip 50\p@
}}
\def\@makeschapterhead#1{
	\vspace*{4\p@}
	{\parindent \z@ \centering \reset@font
		\thickhrulefill
		\par\nobreak
		\vspace*{4\p@}
		\interlinepenalty\@M
		\hrule
		\vspace*{4\p@}
		\Huge \bfseries #1\par\nobreak
		\vspace*{4\p@}
		\hrule
		\vskip 50\p@
}}

\pagestyle{fancy}
\fancyhead{}
\fancyfoot{}
\fancyhead[L]{\slshape\MakeUppercase{\textbf{RASD}}}
\fancyhead[R]{\slshape{Morreale,Maddes,Innocente}}
\fancyfoot[C]{\thepage}
\renewcommand{\footrulewidth}{1pt}
\renewcommand{\headrulewidth}{1pt}
\linespread{1.3}
%\floatstyle{boxed}
\restylefloat{figure}
\parindent 0ex
%\renewcommand{baselinestretch}{1.5}

%%%%%%%%%%%%

% COMMANDS
\newcommand\requirement[1]{\item[{[R#1]}] }
\newcommand\goal[1]{\item[{[G#1]}] }
\newcommand\assumption[1]{\item[{[D#1]}] }
\newcommand\usecase[1]{ [UC#1] }

%%%%%%%%%%%%

%BODY
\begin{document}
	\begin{titlepage}
		\centering
		\vspace*{0.7 cm}
		\includegraphics[scale = 0.85]{assets/polimi.png}\\[1.6 cm]
		\textsc{\large Department of Computer Science and Engineering}\\[1.8 cm]
		
		\rule{\linewidth}{0.2 mm} \\[0.4 cm]
		{ \huge \bfseries Requirement Analysis and Specification Document (RASD)}\\
		\rule{\linewidth}{0.2 mm} \\[1.5 cm]
		
		\textsc{\Large Safe Streets}\\[0.5 cm]
		\textsc{\large - v1.0 -}\\[1 cm]
		
		\begin{minipage}{0.4\textwidth}
			\begin{flushleft} \large
				\emph{Authors:}\\
				\textbf{Morreale} Federico \\
				\textbf{Maddes} Evandro \\
				\textbf{Innocente} Federico
			\end{flushleft}
		\end{minipage}~
		\begin{minipage}{0.4\textwidth}
			\begin{flushright} \large
				%\emph{Student Number:} \\
				945238 \\
				000000 \\
				000000
			\end{flushright}
		\end{minipage}\\[2 cm]
		
		{\large November \nth{11} , 2019}\\[2 cm]
		
		\vfill
	\end{titlepage}
	
	\pagenumbering{roman}
	\tableofcontents
%	\thispagestyle{empty}
	\newpage
	%\listoffigures
	%\listoftables
%	\thispagestyle{empty}
	\clearpage
	\pagenumbering{arabic}
	\setcounter{page}{1}
	
	\chapter{Introduction}
	\section{Purpose}
	\subsection{General Purpose}
	The purpose of the Requirement Analysis and Specification Document (RASD) is to give a clear and non-ambiguos analysis of SafeStreet, an application that will be implemented, describing every aspects of it like functional/non-functional requirements, constraints, domain assumptions, providing use cases and scenerios of the external world. Moreover, a more formal analysis of some relevant functions of the system will be provided using Alloy, a declarative specification language for expressing complex structural constraints and behavior in a software system.
	SafeStreets is a crowd-sourced application that intends to provide users with the possibility to notify authorities when traffic violations occur, and in particular parking violations.
	\begin{itemize}
		\item \textbf{Basic Service:}
		it allows every user to take a picture of a violation with a brief description and send everything, including date,time, position and license plate, to SafeStreet, that will give access to all the information, with different levels of visiblity, to both citizens and police officers. In addition to that, Safestreet will provide the areas with the highest number of violations and a ranking of the vehicles that commit the major number of infractions.
		\item \textbf{Advanced Function 1:}
		it allows SafeStreet to cross information about accidents that occur on the territory of the municipality with its own, in order to give useful advices by suggesting possible interventions. This service is possible if, and only if the local municipality shares its data to SafeStreet.
		\item \textbf{Advanced Function 2:}
		it allows the municipality to generate traffic tickets from the data stored in SafeStreet, of course that data has to be previously validated by the system, preventing malicious corruptions by offenders. Moreover, having the information about issued tickets, SafeStreet can build statistics and share them among all users.
	\end{itemize}
	\subsection{Goals}
	.
	\section{Scope}
	\subsection{Description of the given problem}
	.
	
	\subsection{World and shared phenomena}
	\begin{itemize}
		\item \textbf{World Phenomena}
		\\
		.
		
		\item \textbf{Shared Phenomena}
		\\
		.
	\end{itemize}
	
	\subsection{Goals}
	.
	
	\section{Definitions, Acronyms, Abbreviations}
	\subsection{Definitions}
	.
	
	\subsection{Acronyms}
	.
	
	\subsection{Abbreviations}
	.
	
	\section{Revision history}
	.
	
	\section{Document structure}
	.

	
	
\end{document}
