\providecommand{\rasd}{..}
\documentclass[../RASD.tex]{subfiles}

\begin{document}
    \chapter{Implementation, integration and test plan}\label{ch:implementation,-integration-and-test-plan}
    In this section there is the order in which we plan to implement the subcomponents of our system and the order in which we plan to integrate such subcomponents and test the integration.
    \section{Component implementation}\label{sec:component-implementation}
    The develop of the system will concern only the UserMobileApp, since the server, the map and the storage will be provided by external companies. The system will be implemented, but also tested and integrated, following a bottom-up strategy.
    The external system’s components will not need to be implemented and tested, since they can be considered reliable. Nonetheless, the interaction with these system will need to be tested as well, to check that it has been properly interfaced with SafeStreets.
    Since the features has different importance in the context of the application, the following table will resume some consideration on each functionality in the context of the application, the user and the development. In particulare, the importance for the customer will consider the impact on the customer itself in his experience inside the application and the reasons that could bring him to join and use SafeStreets; the importance for SafeStreets will consider the possible impact of SafeStreets in the market, the use that it could do to improve the system itself and the possible legal aspects; the implementation difficulty will provide a qualitive estimation of the amount of work and time that will be needed to develop that particular feature.



\end{document}
