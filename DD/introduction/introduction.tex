\providecommand{\rasd}{..}
\documentclass[../RASD.tex]{subfiles}

\begin{document}
    \chapter{Introduction}\label{ch:introduction}
    \section{Purpose}\label{sec:purpose}
    The Design Document (DD) aims to specify in more details the technical, functional description on the system-to-be.
    In particular, it provides an overall guidance to the architecture of the system, it describes the main components,
    their interface and deployment, it also focuses on run-time behavior, design pattern, implementation, integration and test planning.
    In this way, this document is mainly aimed to the developers in fact it represents a guide during the development process.
    \section{Scope}\label{sec:scope}
    With reference to what has been stated in the RASD document, SafeStreets intends to provide users with the possibility
    to notify authorities when traffic violations occur, and in particular parking violations.
    The application allows users to send pictures of violations, including their date, time, and position, to authorities.
    In more details, SafeStreets stores the information provided by users, completing it with suitable meta-data.
    In addition, the application allows both end users and authorities to mine the information that has been received.
    Of course, different levels of visibility could be offered to different roles.
    If the municipality offers a service that allows users to retrieve the information about the accidents that occur on the territory of the municipality,
    SafeStreets can cross this information with its own data to identify potentially unsafe areas, and suggest possible interventions.
    In addition, the municipality could offer a service that takes the information about the violations coming from SafeStreets, and generates traffic tickets from it.
    This addition information about issued tickets can be used by SafeStreets to build statistics.
    \newpage
    \section{Definitions, Acronyms, Abbreviations}\label{sec:definitions,-acronyms,-abbreviations}
    \subsection{Definitions}\label{subsec:definitions}
    \begin{itemize}
        \item \textbf{User:} is a general customer that use the application.
        It is used to refer to both an authority and a citizen.
        \item \textbf{Citizen:} is the basic customer of the application.
        He can upload violations and query the system to get statistics on a selected area.
        \item \textbf{Authority:} advanced customer of the application, is a registered user that is qualified to make fines and view sensitive information.
        Must verify himself during the registration.
        \item \textbf{Municipality:} is always intended as the local municipality.
        Every municipality provides and receives information exactly and only about its own jurisdiction.
        \item \textbf{Violation:} is a general infringement reported by a user.
        \item  \textbf{Dossier:} is the instance of a violation on the system, that includes the pictures, position, classification, textual specification and the mark.
        \item \textbf{Authority database}: is the database in which are stored all data about incidents.
        This data are uploaded by authority while the interface to access the database is offered by the municipality.
        Safestreets uses this interface to retrieve data from the database.
        \item
    \end{itemize}

    \subsection{Acronyms}\label{subsec:acronyms}
    \begin{itemize}
        \item RASD: Requirement Analysis and Specification Document
        \item LP: License Plate
        \item GPS: Global Positioning System
        \item API: Application Programming Interface
        \item UML: Unified Modeling Language
    \end{itemize}

    \subsection{Abbreviations}\label{subsec:abbreviations}
    \begin{itemize}
        \item $[Gn]$: n-goal.
        \item $[Dn]$: n-domain assumption.
        \item $[Rn]$: n-functional requirement.
        \item $[UCn]$: n-use case.
    \end{itemize}
    \newpage
    \section{Revision history}\label{sec:revision-history}
    \section{Document structure}\label{sec:document-structure}
    \textbf{Chapter 1: Introduction.}
    A general introduction and overview of the Design Document.
    It aims giving general but exhaustive information about what this document is going to explain.
    \\
    \textbf{Chapter 2: Architectural design.}
    This section contains an overview of the high level components of the system-to-be and then a more detailed description of three architecture views:
    component view, deployment view and runtime view.
    Finally it shows the chosen architecture styles and patterns.
    \\
    \textbf{Chapter 3: User interface design.}
    This section contains the UX Diagrams of the system to be according to the mockups given in the RASD.
    \\
    \textbf{Chapter 4: Requirements Traceability.}
    This section explains how the requirements defined in the RASD map to the design elements defined in this document.
    \\
    \textbf{Chapter 5: Implementation, integration and test plan.}
    This section identifies the order in which it is planned to implement the sub-components of the system,
    the integration of such sub-components and test the integration.
    \\
    \textbf{Chapter 6: Effort Spent.} A summary of the worked time by each member of the group.
    At the end there are an Appendix and a Bibliography.

\end{document}
