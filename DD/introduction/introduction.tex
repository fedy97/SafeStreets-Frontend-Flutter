\providecommand{\rasd}{..}
\documentclass[../RASD.tex]{subfiles}

\begin{document}
    \chapter{Introduction}\label{ch:introduction}
    \section{Purpose}\label{sec:purpose}
    \newpage
    \section{Scope}\label{sec:scope}
    \newpage
    \section{Definitions, Acronyms, Abbreviations}\label{sec:definitions,-acronyms,-abbreviations}
    \subsection{Definitions}\label{subsec:definitions}
    \begin{itemize}
        \item \textbf{User:} is a general customer that use the application. It is used to refer to both an authority and a ciziten.
        \item \textbf{Citizen:} is the basic customer of the application. He can upload violations and query the system to get statistics on a selected area.
        \item \textbf{Authority:} advanced customer of the application, is a registered user that is qualified to make fines and view sensitive information. Must verify himself during the registration.
        \item \textbf{Municipality:} is always intended as the local municipality. Every municipality provides and receives information exactly and only about its own jurisdiction.
        \item \textbf{Violation:} is a general infringement reported by a user.
        \item  \textbf{Dossier:} is the instance of a violation on the system, that includes the pictures, position, classification, textual specification and fine mark.
    \end{itemize}

    \subsection{Acronyms}\label{subsec:acronyms}
    \begin{itemize}
        \item RASD: Requirement Analysis and Specification Document
        \item LP: License Plate
        \item GPS: Global Positioning System
        \item API: Application Programming Interface
        \item UML: Unified Modeling Language
    \end{itemize}

    \subsection{Abbreviations}\label{subsec:abbreviations}
    \begin{itemize}
        \item $[Gn]$: n-goal.
        \item $[Dn]$: n-domain assumption.
        \item $[Rn]$: n-functional requirement.
        \item $[UCn]$: n-use case.
    \end{itemize}
    \newpage
    \section{Revision history}\label{sec:revision-history}
    \newpage
    \section{Reference Documents}\label{sec:reference-documents}
    This document follows ISO/IEC/IEEE 29148:2011 and IEEE 830:1998 standard for software product specifications.
    All the specifications of this project have been given by Rossi and Di Nitto for the Software Engineering 2 Mandatory Project 2019-2020.
    \newpage
    \section{Document structure}\label{sec:document-structure}
    \textbf{Chapter 1: Introduction.}
    \\
    \textbf{Chapter 2: Architectural design.}
    \\
    \textbf{Chapter 3: User interface design.}
    \\
    \textbf{Chapter 4: requirements traceability.}
    \\
    \textbf{Chapter 5: Implementation, integration and test plan.}
    \\
    \textbf{Chapter 6: Effort Spent.} A summary of the worked time by each member of the group.
    \\
    \textbf{Chapter 7: References.} Some useful documents that we followed in order to produce this document.

\end{document}
