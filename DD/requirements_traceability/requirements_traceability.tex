\providecommand{\rasd}{..}
\documentclass[../RASD.tex]{subfiles}

\begin{document}
    \chapter{Requirements traceability}\label{ch:requirements-traceability}
    Below is shown a mapping between requirements defined in the RASD and the design components in the logic application
    that will ensure their fulfillment in a direct way.
    Other components that are indirectly needed to enforce some requirements are not mentioned in the list,
    but their role has been made clear through some comments:
    \begin{enumerate}
        \requirement{1} Unregistered user cannot use SafeStreets.
        \begin{itemize}
            \item \textbf{Access manager} and authentication provider : with the services offered by authentication provider,
            Access manager checks if a user is registered or not.
        \end{itemize}

        \requirement{2} In signing up, users must provide an email and a password.
        \begin{itemize}
            \item \textbf{Access manager}: it checks if all the fields of the registration are filled in the correct way.
        \end{itemize}

        \requirement{3} In signing up, an authority must also provide an ID\@.
        \begin{itemize}
            \item \textbf{Access manager}: it checks if all the fields of the registration are filled in the correct way.
        \end{itemize}

        \requirement{4} In signing up, users must accept “terms \& conditions”.
        \begin{itemize}
            \item \textbf{Access manger} and authentication provider: in sign up access manager shows “terms \& conditions” to the user.
        \end{itemize}

        \requirement{5} SafeStreets identifies a user by its email.
        \begin{itemize}
            \item \textbf{Access Manager}: each user has an identifier which is the user mail.
        \end{itemize}

        \requirement{6} SafeStreets collects user data in its database.
        \begin{itemize}
            \item \textbf{Access manager} and database provider: for each user access manager stores user mail in the database provider
            to keep an association between the user and his reports.
        \end{itemize}

        \requirement{7} SafeStreets stores all reports in its database.
        \begin{itemize}
            \item \textbf{NewReport manager} and database provider: when user adds a new report, it is stored through the database provider.
        \end{itemize}

        \requirement{8} SafeStreets retrieves reports from its database.
        \begin{itemize}
            \item \textbf{ViolationQuery manager} and database provider: this manager contains all the procedures to make a query on the database
            to retrieve some specific data.
        \end{itemize}

        \requirement{9} SafeStreets authenticates a user when tries to log in.
        \begin{itemize}
            \item \textbf{Access manager} and authentication provider: through the services offered by the authentication provider, access manager logs in a user.
        \end{itemize}

        \requirement{10} Users can view the reports in the map.
        \begin{itemize}
            \item \textbf{ReportMap manager} and map provider: this requirements is the main functionality of this manager.
            It displays all the daily reports on the map.
        \end{itemize}


        \requirement{11} Authorities can see all the sensitive information contained in a report.
        \begin{itemize}
            \item \textbf{UserReportVisualization manager}: this manager, based on the type of user, can show or not the sensible information of one report.
            For the citizens shows the images with the plate blurred.
        \end{itemize}

        \requirement{12} Citizens cannot see sensitive information in a report.
        \begin{itemize}
            \item \textbf{UserReportVisualization manager}: this manager, based on the type of user, can show or not the sensible information of one report.
            For the citizens shows the images with the plate blurred.
        \end{itemize}

        \requirement{13} Users can view unsafe areas in the map.
        \begin{itemize}
            \item \textbf{Unsafe area manager} and map provider: this manager shows unsafe areas on the map archived from the map provider.
        \end{itemize}

        \requirement{14} Users can delete reports they submitted within a day.
        \begin{itemize}
            \item \textbf{UserReportVisualization manager}: this manager manages all the operation on one report, in this case the deletion of a report.
        \end{itemize}

        \requirement{15} Users can edit reports they submitted within a day.
        \begin{itemize}
            \item \textbf{UserReportVisualization manager}: this manager manages all the operation on one report, in this case the editing of a report.
        \end{itemize}

        \requirement{16} Users can upload \textit{valid} reports.
        \begin{itemize}
            \item \textbf{NewReport manager}: this manager allows one user to upload a report.
        \end{itemize}

        \requirement{17} A report must contain at least one image.
        \begin{itemize}
            \item \textbf{NewReport manager}: this manager checks the completeness and the correctness of one report that it’s going to be added.
        \end{itemize}

        \requirement{18} A report must indicate the type of violation.
        \begin{itemize}
            \item \textbf{NewReport manager}: this manager checks the completeness and the correctness of one report that it’s going to be added.
        \end{itemize}

        \requirement{19} A report is \textit{valid} if and only if R17 and R18 are satisfied.
        \begin{itemize}
            \item \textbf{NewReport manager}: this manager checks the completeness and the correctness of one report that it’s going to be added.
        \end{itemize}

        \requirement{20} Images containing sensitive information (like license plates) must be blurred by the system.
        \begin{itemize}
            \item \textbf{NewReport manager} and plate recognizer provider: this manager, using the plate recognizer provider,
            is able to get the pixels of the plates on the images uploaded with a report.
        \end{itemize}

        \requirement{21} The system must notify the user if the report submitted is not valid.
        \begin{itemize}
            \item \textbf{NewReport manager}: this manager checks the completeness and the correctness of one report that it’s going to be added.
            If it does not respect all the constraints, the user will be notified.
        \end{itemize}

        \requirement{22} For each type of statistics there must be at least one user from which SafeStreets takes data.
        \begin{itemize}
            \item \textbf{Statistics manager}: this manager checks all the conditions to build one type of statistic.
        \end{itemize}

        \requirement{23} A user can access to feedback area for each report.
        \begin{itemize}
            \item \textbf{UserReportVisualization manager}: this manager manages all the operation on one report, in this case the feedback area of a report.
        \end{itemize}

        \requirement{24} A user can select negative feedback for each report.
        \begin{itemize}
            \item \textbf{UserReportVisualization manager}: this manager manages all the operations on one report, in this case the feedback area of a report.
        \end{itemize}

        \requirement{25} Authorities can see suggestions for possible interventions.
        \begin{itemize}
            \item \textbf{UnsafeArea manager}: for each unsafe area, this manager suggests possible interventions to authorities.
            The suggestions are based on the type of unsafe area (type of most accidents committed).
        \end{itemize}

        \requirement{26} During adding a new report, if there is a report of same type, in the same position and at the same date,
        SafeStreets shows the report stored to the user if the violation is the same.
        \begin{itemize}
            \item \textbf{NewReport manager} and database provider: this manager checks the correctness of a new report.
            \item \textbf{ViolationQuery manager} and database provider: this manager checks if a report that is going to be added has the same position
            and data of at least one report already stored in the database.
        \end{itemize}

        \requirement{27} Authorities can mark a report as “fined”.
        \begin{itemize}
            \item \textbf{UserReportVisualization manager}: this manager manages all the operation on one report, in this case the specific “fined field”
            of a report that is available only for the authorities.
        \end{itemize}

        \requirement{28} User can query SafeStreets to achieve a specific report.
        \begin{itemize}
            \item \textbf{ViolationQuery manager} and database provider: Users have a dedicated area in the application where they can make a query
            by setting different fields (data, type, zone).
        \end{itemize}

        \requirement{29} SafeStreets retrieves incidents from municipality database.
        \begin{itemize}
            \item \textbf{Statistics manager} and database authority: It retrieves data using database of the authority in which are stored data about accidents.
        \end{itemize}
    \end{enumerate}

    \newpage
    %%traceability matrix
    \begin{adjustwidth}{+2cm}{}
        \begin{longtable}[H]
        {|| p{.02\linewidth} || p{.12\linewidth} | p{.12\linewidth} | p{.12\linewidth} | p{.13\linewidth} | p{.12\linewidth} |
        p{.19\linewidth} | p{.13\linewidth} |}
            \hline
            \textbf{\makecell{R}} & \textbf{\makecell{Access \\ manager}} & \textbf{\makecell{New \\Report \\ manager}} & \textbf{\makecell{Report \\Map \\ manager}} & \textbf{\makecell{Statistics \\ manager}} &
            \textbf{\makecell{Unsafe \\ Area \\ manager}} & \textbf{\makecell{User \\ Report \\ Visualization \\ manager}} & \textbf{\makecell{Violation \\ Query \\ manager}}\\ \hline
            \textbf{1} & \textit{Yes} & & & & & & \\ \hline
            \textbf{2} & \textit{Yes} & & & & & & \\ \hline
            \textbf{3} & \textit{Yes} & & & & & & \\ \hline
            \textbf{4} & \textit{Yes} & & & & & & \\ \hline
            \textbf{5} & \textit{Yes} & & & & & & \\ \hline
            \textbf{6} & \textit{Yes} & & & & & & \\ \hline
            \textbf{7} & & \textit{Yes} & & & & & \\ \hline
            \textbf{8} & & & & & & & \textit{Yes}\\ \hline
            \textbf{9} & \textit{Yes} & & & & & & \\ \hline
            \textbf{10} & & & \textit{Yes} & & & & \\ \hline
            \textbf{11} & & & & & & \textit{Yes} & \\ \hline
            \textbf{12} & & & & & & \textit{Yes} & \\ \hline
            \textbf{13} & & & & & \textit{Yes} & & \\ \hline
            \textbf{14} & & & & & & \textit{Yes} & \\ \hline
            \textbf{15} & & & & & & \textit{Yes} & \\ \hline
            \textbf{16} & & \textit{Yes} & & & & & \\ \hline
            \textbf{17} & & \textit{Yes} & & & & & \\ \hline
            \textbf{18} & & \textit{Yes} & & & & & \\ \hline
            \textbf{19} & & \textit{Yes} & & & & & \\ \hline
            \textbf{20} & & \textit{Yes} & & & & & \\ \hline
            \textbf{21} & & \textit{Yes} & & & & & \\ \hline
            \textbf{22} & & & & \textit{Yes} & & & \\ \hline
            \textbf{23} & & & & & & \textit{Yes} & \\ \hline
            \textbf{24} & & & & & & \textit{Yes} & \\ \hline
            \textbf{25} & & & & & \textit{Yes} & & \\ \hline
            \textbf{26} & & \textit{Yes} & & & & & \textit{Yes} \\ \hline
            \textbf{27} & & & & & & \textit{Yes} & \\ \hline
            \textbf{28} & & & & & & & \textit{Yes}\\ \hline
            \textbf{29} & & & & \textit{Yes} & & & \\ \hline
            \caption[\textit{Traceability matrix}]{\textit{Traceability matrix}}
        \end{longtable}
    \end{adjustwidth}
\end{document}
