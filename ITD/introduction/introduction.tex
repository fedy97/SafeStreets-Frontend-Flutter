\providecommand{\rasd}{..}
\documentclass[../RASD.tex]{subfiles}

\begin{document}
    \chapter{Introduction}\label{ch:introduction}
    \section{Purpose}\label{sec:purpose}
    This Document contains a detailed and exhaustive explanation of the implementation of the SafeStreets project.
    The purpose of this document is to explicitly detail the rationale of all choices made and provide a complete overview of how the project source code is structured and designed.
    The document also tackles the software testing activities and shows what are the main tests that have been performed on the product.
    \section{Scope}\label{sec:scope}
    This  document  is  a  follow  up  on  the  previous  two,  \textit{Requirement  analysis  and  Specification document} and \textit{Design Document}:  it will describe the prototype for SafeStreets system that has been developed in accordance to them.
    It's important to remember that SafeStreets intends to provide users with the possibility to notify authorities when traffic violations occur, and in particular parking violations.
    The application allows users to send pictures of violations, including their date, time, and position, to authorities.
    In more details, SafeStreets stores the information provided by users, completing it with suitable meta-data.
    In addition, the application allows both end users and authorities to mine the information that has been received.
    Of course, different levels of visibility could be offered to different roles.
    If the municipality offers a service that allows users to retrieve the information about the accidents that occur on the territory of the municipality,
    SafeStreets can cross this information with its own data to identify potentially unsafe areas, and suggest possible interventions.
    In addition, the municipality could offer a service that takes the information about the violations coming from SafeStreets, and generates traffic tickets from it.
    This addition information about issued tickets can be used by SafeStreets to build statistics.
    \newpage
    \section{Definitions, Acronyms, Abbreviations}\label{sec:definitions,-acronyms,-abbreviations}
    \subsection{Definitions}\label{subsec:definitions}
    \begin{itemize}
        \item \textbf{User:} is a general customer that use the application.
        It is used to refer to both an authority and a citizen.
        \item \textbf{Citizen:} is the basic customer of the application.
        He can upload violations and query the system to get statistics on a selected area.
        \item \textbf{Authority:} advanced customer of the application, is a registered user that is qualified to make fines and view sensitive information.
        Must verify himself during the registration.
        \item \textbf{Municipality:} is always intended as the local municipality.
        Every municipality provides and receives information exactly and only about its own jurisdiction.
        \item \textbf{Violation:} is a general infringement reported by a user.
        \item  \textbf{Dossier:} is the instance of a violation on the system, that includes the pictures, position, classification, textual specification and the mark.
        \item \textbf{Authority database}: is the database in which are stored all data about incidents.
        This data are uploaded by authority while the interface to access the database is offered by the municipality.
        SafeStreets uses this interface to retrieve data from the database.
        \item
    \end{itemize}

    \subsection{Acronyms}\label{subsec:acronyms}
    \begin{itemize}
        \item RASD: Requirement Analysis and Specification Document
        \item LP: License Plate
        \item GPS: Global Positioning System
        \item API: Application Programming Interface
        \item UML: Unified Modeling Language
    \end{itemize}

    \subsection{Abbreviations}\label{subsec:abbreviations}
    \begin{itemize}
        \item $[Gn]$: n-goal.
        \item $[Dn]$: n-domain assumption.
        \item $[Rn]$: n-functional requirement.
        \item $[UCn]$: n-use case.
    \end{itemize}
    \newpage
    \section{Revision history}\label{sec:revision-history}
    \begin{itemize}
        \item V1.0, January \nth{12} 2020: First release
    \end{itemize}
    \section{Document structure}\label{sec:document-structure}
    \textbf{Chapter 1: Introduction.}
    A general introduction and overview of the \textit{Implementation and testing document}.
    It aims giving general but exhaustive information about what this document is going to explain.


\end{document}
