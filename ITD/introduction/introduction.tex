\providecommand{\rasd}{..}
\documentclass[../RASD.tex]{subfiles}

\begin{document}
    \chapter{Introduction}\label{ch:introduction}
    \section{Purpose}\label{sec:purpose}

    \section{Scope}\label{sec:scope}

    \newpage
    \section{Definitions, Acronyms, Abbreviations}\label{sec:definitions,-acronyms,-abbreviations}
    \subsection{Definitions}\label{subsec:definitions}
    \begin{itemize}
        \item \textbf{User:} is a general customer that use the application.
        It is used to refer to both an authority and a citizen.
        \item \textbf{Citizen:} is the basic customer of the application.
        He can upload violations and query the system to get statistics on a selected area.
        \item \textbf{Authority:} advanced customer of the application, is a registered user that is qualified to make fines and view sensitive information.
        Must verify himself during the registration.
        \item \textbf{Municipality:} is always intended as the local municipality.
        Every municipality provides and receives information exactly and only about its own jurisdiction.
        \item \textbf{Violation:} is a general infringement reported by a user.
        \item  \textbf{Dossier:} is the instance of a violation on the system, that includes the pictures, position, classification, textual specification and the mark.
        \item \textbf{Authority database}: is the database in which are stored all data about incidents.
        This data are uploaded by authority while the interface to access the database is offered by the municipality.
        SafeStreets uses this interface to retrieve data from the database.
        \item
    \end{itemize}

    \subsection{Acronyms}\label{subsec:acronyms}
    \begin{itemize}
        \item RASD: Requirement Analysis and Specification Document
        \item LP: License Plate
        \item GPS: Global Positioning System
        \item API: Application Programming Interface
        \item UML: Unified Modeling Language
    \end{itemize}

    \subsection{Abbreviations}\label{subsec:abbreviations}
    \begin{itemize}
        \item $[Gn]$: n-goal.
        \item $[Dn]$: n-domain assumption.
        \item $[Rn]$: n-functional requirement.
        \item $[UCn]$: n-use case.
    \end{itemize}
    \newpage
    \section{Revision history}\label{sec:revision-history}
    \begin{itemize}
        \item V1.0, January \nth{12} 2020: First release
    \end{itemize}
    \section{Document structure}\label{sec:document-structure}
    \textbf{Chapter 1: Introduction.}
    A general introduction and overview of the Design Document.
    It aims giving general but exhaustive information about what this document is going to explain.
    

\end{document}
