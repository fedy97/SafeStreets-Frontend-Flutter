\providecommand{\rasd}{..}
\documentclass[../RASD.tex]{subfiles}

\begin{document}
    \chapter{Installation Instructions}\label{ch:installation-instructions}
\section{Flutter Application Setup}\label{sec:flutter-application-setup}
In this section will be present all the necessary information to build the
Flutter project.
\subsection{Install Flutter Framework}\label{subsec:install-flutter-framework}
Detailed installation instructions of the framework depending on the operating system (Windows, MacOS, Linux)
    can be found following
    \href{https://flutter.dev/docs/get-started/install}{\textit{this}} link.
    our team did not test flutter on Windows.
\subsection{Install Plugins on Android Studio}\label{subsec:install-plugins-on-android-studio}
    Flutter runs on both Intellij and Android Studio IDE, so you can choose which one to use.
    Android studio can be found \href{https://developer.android.com/studio}{\textit{here}}.
    Once you have installed it, you have to download the plugins for Flutter and Dart (that is the programming language on which flutter is built).
    In order to do that:
    \begin{enumerate}
        \item Start Android Studio.
        \item Open plugin preferences (Preferences > Plugins on macOS, File > Settings > Plugins on Windows and Linux).
        \item Select Marketplace, select the Flutter plugin and click Install.
        \item Click Yes when prompted to install the Dart plugin.
        \item Click Restart when prompted.
    \end{enumerate}
\subsection{Install Plugins on IntelliJ}\label{subsec:install-plugins-on-intellij}
    IntelliJ can be found \href{https://www.jetbrains.com/idea/download/}{\textit{here}}.
    you need version 2017.1 or later.
    Once you have installed it, you have to download the plugins for Flutter and Dart (that is the programming language on which flutter is built).
    In order to do that:
    \begin{enumerate}
        \item Start IntelliJ.
        \item Open plugins tab (File > Settings > Plugins).
        \item Select Marketplace, search for the Flutter plugin and click Install.
        \item Click Yes when prompted to install the Dart plugin.
        If it does not ask it, install it manually.
        \item Click Restart when prompted.
    \end{enumerate}
\subsection{Install Android Emulator on Android Studio}\label{subsec:install-android-emulator-on-android-studio}
    Our team used IntelliJ, so we leave the installation instructions
    \href{https://docs.expo.io/versions/latest/workflow/android-studio-emulator/}{\textit{here}}.
    a FULL-HD screen resolution device is recommended(1080x1920).
\subsection{Install Android Emulator on IntelliJ}\label{subsec:install-android-emulator-on-intellij}
    On IntelliJ:
    \begin{enumerate}
        \item Go to Tools > Android > AVD Manager
        \item Click on Create Virtual Device
        \item Select a Device (a FULL-HD screen resolution device is recommended(1080x1920))
        \item Click Next
        \item Download Pie or Q from the list, then click Next
        \item Click Finish
    \end{enumerate}
    You have now installed android emulator.
\subsection{Use your own Android phone}\label{subsec:use-your-own-android-phone}
    Alternatively, you can use your own smartphone to test the app, you have to activate debug usb from
    the developer options of your device and have adb services installed on your PC.

\subsection{Get Project Dependencies}\label{subsec:get-project-dependencies}
    Once you have done everything above, you can now navigate to `MorrealeMaddesInnocente
    /safestreets/pubspec.yaml'
    and click on packages get on the top right (if you are using Intellij), or simply open the terminal and run
    `flutter packages get'.
    If something goes wrong, run `flutter doctor' in the terminal.
\subsection{Run the Application}\label{subsec:run-the-application}
    Open the android emulator (or connect your device) and then go to\newline
    `MorrealeMaddesInnocente/safestreets/lib/main.dart` and click on the green arrow.
    If you do not find the arrow, go to Edit Configurations and click on the `+` top left,
    then click Flutter and in the `Dart Entrypoint' you select the file main.dart.
    Now you can run the App, the first run will take longer than usual, don't worry.
\end{document}