\providecommand{\atd}{..}
\documentclass[../ATD.tex]{subfiles}

\begin{document}
    \chapter{Installation setup}\label{ch:installation-setup}
    \section{References}\label{sec:references}
    The installation setup has been done by following the two documents provided by the developer of the system:
    \begin{itemize}
        \item SoftwareToInstall
        \item ITD1
    \end{itemize}
    Both the documents can be found on github, in the folder DeliveryFolder: \href{https://github.com/gianfi12/AbboAccordiBonetti/DeliveryFolder}.

    \section{Installation choice}\label{sec:installation-choice}
    The following software has been used:
    \begin{itemize}
        \item \textbf{Glassfish5}: Glassfish has been used in his 5.0 version instead of the 4.0 (required in the ITD) because of an error that did not allowed to run the server
        The developer group ensure us that this choice was fine, since some of their team worked with the same software.
        \item \textbf{IntelliJ}: we used mainly IntelliJ to run the application, but some tests has been done even with AndroidStudio.
    \end{itemize}

    \section{Installation issues}\label{sec:installation-issues}
    We present here the issues that came up within the installation process.
    \subsection{Server deploy}\label{subsec:server-deploy}
    During the server deploy process, an issue that did not allow the application to communicate with the server came up.
    We have been able to solve it by communicating with the developer team.
    The problem that we found was that the SafeStreetsSOAP and SafeStreetsWebServer were assigned to an unrecognized link, different from the once that were expected.
    The expected links were \textit{\${com.sun.aas.instanceRootURI}/applications/SafeStreetsSOAP/} and \textit{\${com.sun.aas.instanceRootURI}/applications/SafeStreetsWebServer/},
    while we found \textit{\${com.sun.aas.instanceRootURI}/applications/SafeStreetsSOAP534859734543/} and \textit{\${com.sun.aas.instanceRootURI}/applications/SafeStreetsWebServer54378953479583/}.
    To solve the problem, we needed to go in the glassfish5/glassfish/domains/domain1/config folder and modify the domain.xml file by removing the unexpected characters wherever they show up.


\end{document}